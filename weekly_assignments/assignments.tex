\documentclass{homework}
\title{Weekly Problems}
\course{MAT3400}
\author{Ivar Stangeby}
\begin{document}
  \week{9}
  \problem{2.2}
  \begin{problemtext}
    Let $S$ be any non-empty set and let $X$ be a normed space over $\mathbb{F}$. Let $F_b(S, X)$ be the linear subspace of $F(S, X)$ of all functions $f : S \to X$ such that $\left\{ \|f(s)\|_{b} \mid s \in S \right\}$ is bounded. Show that $F_b(S, X)$ has a norm defined by
    \begin{equation}
      \notag
      \|f\|_{b} = \sup \left\{ \|f(s)\|_{}\mid s \in S \right\}.
    \end{equation}
  \end{problemtext}

  \begin{solution}
    First recall that $\|\cdot\|_{}$ is the norm on $X$, and consequently on
    $S$.  Firstly, we see that $\sup \left\{ \|f(s)\|_{} \right\} \geq 0$ for
    all $s \in S$ with $\|f(s)\|_{} = 0$ if and only if $f = 0$.  Let $\alpha
    \in \mathbb{F}$. Then $\sup \left\{ \|\alpha f(s)\|_{} \right\} = \alpha
  \sup \left\{ \|f(s)\|_{} \right\}$. Lastly, let $f, g \in F_b(S, X)$. Then
\begin{equation} \notag \sup \left\{ \|f + g\|_{} \right\} \leq \sup \left\{
\|f\| \right\} + \left\{ \|g\| \right\} \end{equation} It then follows that
$\|\cdot\|_{b}$ is a norm on $F_b(S, X)$.
  \end{solution}
  \problem{2.3}

  \begin{problemtext}
    For each $n \in \mathbb{N}$ let $f_n : \left[ 0, 1 \right] \to \mathbb{R}$ be defined by $f_n(x) = x^n$. Find the norm of $f_n$ in the following cases:
    \begin{enumerate}[a)]
      \item in the normed space $C_\mathbb{R}\left( \left[ 0, 1\right] \right)$;
      \item in the normed space $L^1[0, 1]$.
    \end{enumerate}
  \end{problemtext}
  \begin{solution}\hfill
    \begin{enumerate}[a)]
      \item The \emph{standard norm} on $C_\mathbb{R}\left( \left[ 0,1 \right] \right)$ is defined by
        \begin{equation}
          \notag
          \|f\|_{} = \sup \left\{ |f(x)| \mid x \in \left[ 0, 1 \right] \right\}.
        \end{equation}
        Therefore, $\|x^n\|_{} = 1$.
      \item The standard norm on $L^1[0, 1]$ is defined by
        \begin{equation}
          \notag
          \|f\|_{1} = \int_0^1 |f(x)| \, dx.
        \end{equation}
        Then $\|x^n\|_{1} = \int_0^1 |x^n| \, dx = \frac{1}{n+1}$.
    \end{enumerate}
  \end{solution}
  \problem{2.12}

  \begin{problemtext}
    Let $X$ be a normed linear space and, for any $x \in X$ and $r >0$, let $T = \left\{ y \in X \mid \|y-x\|_{} \leq r \right\}$ and $S = \left\{ y \in X \mid \|x - y\|_{} < r \right\}$.
    \begin{enumerate}[a)]
      \item Show that $T$ is closed.
      \item If $z \in T$ and $z_n = \left( r-n^{-1} \right)z$, for $n \in
        \mathbb{N}$, show that $\lim_{n\to\infty} z_n = z$ and hence show that
        $\overline{S} = T$.\unsure{The proposed solution uses $z_n = \left( 1 -
        n^{-1} \right)z$ so I am going to assume that there is an error in the
      problem text.}
    \end{enumerate}
  \end{problemtext}
  \begin{solution}
    \begin{enumerate}[a)]\hfill
      \item Let $\left\{ y_n \right\}$ be a sequence in $T$ that converge to $z
        \in X$. Then by definition $\|y_n - x\|_{} \leq r$. Taking limits and
        applying Theorem 2.11 we have that
        \begin{equation}
          \notag
          \lim_{n\to\infty} \|y_n - x\|_{} = \|z - x\|_{} \leq r
        \end{equation}
        hence $z \in T$ and consequently $T$ is closed.
      \item We have that
      \begin{equation}
        \notag
        \|z_n - z\|_{} = \|\left( (1 - n^{-1})z - z \right)\|_{} =
        n^{-1}\|z\|_{} = n^{-1}\|z - 0\|_{} \leq n^{-1}
      \end{equation}
      hence $\lim_{n\to\infty} z_n = z$.

      Since $S \subseteq T$ and $T$ is closed, $\overline{S} \subseteq T$.
      Conversely, if $z \in T$ and $z_n$ defined as above then $\|z_n\|_{} =
      \left( 1 - n^{-1} \right)\|z_n\|_{} \leq (1 - n^{-1})r < r$. Therefore
      $z$ is a limit of a sequence of elements of $S$, so $z \in \overline{S}$.
      Hence $T \subseteq \overline{S}$ so $T = \overline{S}$.
    \end{enumerate}
  \end{solution}
  \problem{2.14}
  \begin{problemtext}
    Let $S$ be any non-empty set, let $X$ be a Banach space over $\mathbb{F}$
    and let $F_b(S, X)$ be the vector space given in Problem 2.2 with the norm
    $\|f\|_{b} = \sup \left\{ \|f(s)\|_{}\mid s \in S \right\}$. Show that
    $F_b(S, X)$ is a Banach space.
  \end{problemtext}
  \begin{solution}
    Let $\left\{ f_n \right\}$ be Cauchy in $F_b(S, X)$. Then there exists an $N$ such that
    \begin{equation}
      \notag
      \|f_n(s) - f_m(s)\|_{} \leq \|f_n - f_m\|_{b} \leq \varepsilon.
    \end{equation}
    It then follows that $\left\{ f_n(s) \right\}$ is Cauchy in $X$, and since
    $X$ is complete it is a convergent sequence. It is then meaningful to talk of the limit of this sequence, which we denote $f(s) = \lim_{n\to\infty} f_n(s)$. From the Cauchy property we have that
    \begin{equation}
      \notag
      \lim_{m\to\infty} \|f_n(s) - f_m(s)\|_{} = \|f_n(s) - f(s)\|_{} \leq \varepsilon.
    \end{equation}
    Applying the inverse triangle inequality yields
    \begin{equation}
      \notag
      \|f(s)\|_{} \leq \varepsilon+ \|f_n(s)\|_{} \leq \|f_n\|_{b} + \varepsilon.
    \end{equation}
    Hence $\|f(s)\|$ is bounded and we can therefore conclude that $f \in
    F_b(S, X)$.\comment{The reason we did all this was to make sure that the
    limit function was in fact inside the space $F_b(S, X)$.} It then follows
    that $F_b(S, X)$ is a Banach space.
  \end{solution}
  \problem{3.4}
  \begin{problemtext}
    Give the proof of Lemma 3.14.
  \end{problemtext}
  \begin{solution}
    Lemma 3.14 states that if $X$ is an inner product space with inner product
    $\left( \cdot \mid \cdot \right)$. Then for all $u, v, x, y \in X$:
    \begin{enumerate}[a)]
      \item $\left( u + v \mid x + y \right) - \left( u - v \mid x - y \right)
        = 2\left( u \mid y \right) + 2\left( v \mid x \right)$;
      \item $4\left( u \mid y \right) = \left( u+v \mid x+y \right) - \left(
        u-v \mid x-y \right) +i\left( u-iv \mid x-iy \right) -i\left( u-iv \mid
        x-iy \right)$.\\ (for complex $X$).
    \end{enumerate}
    For the first one, we simply perform the calculations, using the linearity
    in the first variable and the conjugate linearity in the second variable.
    For the second we simply observe that the right hand side closely resembles
    the identity in a) so we apply that and use the conjugate linearity in the
    second variable.  We have that
    \begin{align*}
      &\left( u+v  \mid  x+y \right) - \left( u-v  \mid  x-y \right) = 2\left( u  \mid  y \right) + 2\left( v  \mid  x \right) \\
      &i\left( u-iv  \mid  x-iy \right) -i\left( u-iv  \mid  x-iy \right) = -i^2(2\left( u  \mid  y \right) + 2\left( u \mid  x \right) )
    \end{align*}
    Adding these two together yields $4\left( u  \mid  y \right)$ as we wanted
    to show.
  \end{solution}

  \week{10}

  \problem{3.10}

  \problem{3.15}
  \problem{3.22}
  \problem{3.24}
  \problem{3.27}
  \problem{3.28}

  \week{11}
  \problem{6.2}
  \begin{problemtext}
  	Find the adjoint of the linear operator $T : \ell^2 \to \ell^2$ defined by
	\begin{equation}
		\notag
		T \left( x_1, x_2, x_3, x_4, \ldots \right) = \left( 0, 4x_1, x_2, 4x_3, x_4, \ldots \right).
	\end{equation}
  \end{problemtext}
  \begin{solution}
	Theorem 6.1 tells us that an unique adjoint $T^*$ must exist. By the
	property of $T^*$ we have that 
	\begin{equation}
		\notag
		\left( T \left\{ x_n \right\} \mid
		\left\{ y_n \right\}\right) = \left( \left\{ x_n\right\}  \mid \left\{ z_n \right\}  \right)
	\end{equation}
	with $\left\{ z_n \right\} = \left\{ Ty_n \right\}$. This can only be
	true in the case where $z_1 = 4y_2$, $z_2 = y_3$, $z_3 = 4y_4$, and so
	on. Hence, the adjoint $T^*$ of $T$ is defined uniquely through
	\begin{equation}
		\notag
		T^*\left( \left\{ y_n \right\} \right) = \left( 4y_2, y_3, 4y_4, \ldots \right).
	\end{equation}
  \end{solution}

  \problem{6.16}
  \begin{problemtext}
      Let $T \in \mathcal{B}(\ell^2)$ be defined by
      \begin{equation}
          \notag
          T \left( x_1, x_2, x_3, x_4, \ldots \right) = \left( x_1, -x_2, x_3, -x_4, \ldots \right).
      \end{equation}
      \begin{enumerate}[a)]
          \item Show that 1 and -1 are eigenvalues of $T$ with eigenvectors
              $\left( 1, 0, 0, \ldots \right)$ and $\left( 0, 1, 0, 0, \ldots
              \right)$ respectively.
          \item Find $T^2$ and hence show that $\sigma(T) = \left\{ 1, -1 \right\}$.
    \end{enumerate}
  \end{problemtext}
  \begin{solution}
      For a) we simply plug in the values:
      \begin{align*}
          T \left( 1, 0, 0, \ldots \right) &= 1\left( 1, 0, 0, \ldots \right) \\
          T \left( 0, 1, 0, \ldots \right) &= -1\left( 0, 1, 0, \ldots \right) \\
      \end{align*}
      Hence 1 and -1 are eigenvalues with respective eigenvectors.

      For b) we compute $T^2$:
      \begin{equation}
          \notag
          T^2x = TTx^2 = \left( x_1, x_2, x_3, x_3, x_4, \ldots \right)
      \end{equation}
      so $T^2 = I$. Consequently, $\sigma(T^2) = \left\{ 1 \right\}.

  \end{solution}
\end{document}
